\documentclass[]{article}

\usepackage{hyperref}

\begin{document}

%First is our Title

\title{Pete the Platformer}
\author{Oscar Holloway, Jon Rosin Jr.}
\maketitle

%Second is our Table of Contents

\tableofcontents

%Third is our introduction

\section{Introduction}
	This project has taken on various forms over the course of the semester. At first, we were attempting to create a 2D platformer. Eventually, we moved onto creating a 3rd Person game with heavy puzzle and platforming mechanics. Through various forms and shapes, Pete the Platformer is still trying to find its way into existence. 100s of man hours while learning the program at the same time has been interesting and extremely challenging. This is a project I have sunk much of my time into, and I plan on working on this project for years to come. 

\section{Agile Manifesto}
	The four pillars of the Agile Manifesto came into play in nearly every aspect of our development process so far. 'Individuals and Interactions over processes and tools' has been the most important pillar to our team thus far. If you strip everything away, creating software is just processes and tools, but to become passionate about creating software, we need to see the human side of each of these processes. Being involved with our team on characters that we cared about makes me, as a human with emotions, much more involved and interested in the project as a whole.  The pillar we have struggled with the most i 'working software over comprehensive documentation.' As you can see here, I'm somewhat capable at producing comprehensive documentation. With all of our issues with github and getting our game off our machine, we were never able to produce working software in our presentation environment. I do believe, however, that we responded to change well when faced with adversity. Our project has had many faces, and with each one we've learned something different and important. Attempting to satisfy our customer was our primary concern from the beginning, and I believe we welcomed change with open arms. Our environment wasn't pristine considering our limitations, but I believe we did a considerable amount with what we had. 
	
\section{Pete Started as a Platformer}
	When  we were first familiarizing ourselves with Unreal Engine 4, we didn't have much to go on as far as ideas. We thought we would start with a platformer in order to become experienced using the tools at hand. After creating a couple of flipbooks for our character's running and jumping animation, we moved onto creating intelligent platforms within our first level. We also never created a true falling animation, which is still a primary goal of mine going forward. As far as moving platforms go, we were able to get one functioning platform that would move up and down on the screen. It was around this time that our idea of the game was changing. Our ideas were all coming out as puzzles; puzzles that wouldn't be interesting or challenging within a 2D space. This is when we decided to change our game completely. So we started from scratch with a Third-Person 3D template. 
	
\section{3D Pete}
	Where were now in our project taught me the most important lesson I learned this semester: Be flexible. This were changing. Old work was being thrown out the window, and it was time to recreate Pete in a 3D space. Once we had began creating blueprints for light switches that would be involved in our first puzzle, we knew we had done the correct thing in changing our project. What's great is that our creativity involving actual platforming became significantly better once we had a 3D space to work with. We could go higher, wider and ideas got more complicated for the better. Within three days to our final due date, we knew it was time to get this project out the door. This is when everything fell apart. Our inexperience with source code management and our neglect to study it would be our true demise. Well, not really, but we really wanted the rest of the class to see what we had been working on for so long. Unfortunately, that didn't happen.   

\section{Where do We go From Here}
	The goal is to keep improving Pete within my spare time, and hopefully next semester I can have something to truly show for all our hours behind closed doors. My passion since I was an 8th grade student was to one day create video games, and in some respects I'm at that point. I couldn't be more grateful learn about free engines that anyone can create in if they put their mind to it. I plan on tinkering around with Unity and Amazon Lumberyard when I get the chance, but my main focus is Pete the Platformer. I hope to one day release Pete on Steam and hopefully consoles as well. The time I have spent with Unreal has enlightened me a great deal on how creating full-scale Triple A games actually takes place. There's so much that goes on behind the scenes and it requires every member of a team doing alot of not so glamorous work to put a game together. This appreciation will be invaluable for me going forward, and I feel this is the starting point for projects and tools that I will be using for the rest of my career. 

\end{document}
